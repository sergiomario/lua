	% ------------------------------------------------------------------------
% ------------------------------------------------------------------------
% abnTeX2: Modelo de Trabalho Academico (tese de doutorado, dissertacao de
% mestrado e trabalhos monograficos em geral) em conformidade com 
% ABNT NBR 14724:2011: Informacao e documentacao - Trabalhos academicos -
% Apresentacao
% ------------------------------------------------------------------------
% ------------------------------------------------------------------------

\documentclass[
	% -- opções da classe memoir --
	12pt,				% tamanho da fonte
	openright,			% capítulos começam em pág ímpar (insere página vazia caso preciso)
	twoside,			% para impressão em verso e anverso. Oposto a oneside
	a4paper,			% tamanho do papel. 
	% -- opções da classe abntex2 --
	%chapter=TITLE,		% títulos de capítulos convertidos em letras maiúsculas
	%section=TITLE,		% títulos de seções convertidos em letras maiúsculas
	%subsection=TITLE,	% títulos de subseções convertidos em letras maiúsculas
	%subsubsection=TITLE,% títulos de subsubseções convertidos em letras maiúsculas
	% -- opções do pacote babel --
	english,			% idioma adicional para hifenização
	brazil,				% o último idioma é o principal do documento
	]{abntex2}


% ---
% PACOTES
% ---

% ---
% Pacotes fundamentais 
% ---
\usepackage{cite}
\usepackage[brazil]{babel}
\usepackage{cmap}				% Mapear caracteres especiais no PDF
\usepackage{lmodern}			% Usa a fonte Latin Modern			
\usepackage[T1]{fontenc}		% Selecao de codigos de fonte.
\usepackage[utf8]{inputenc}		% Codificacao do documento (conversão automática dos acentos)
\usepackage{lastpage}			% Usado pela Ficha catalográfica
\usepackage{indentfirst}		% Indenta o primeiro parágrafo de cada seção.
\usepackage{color}				% Controle das cores
\usepackage{graphicx}			% Inclusão de gráficos
		
% ---
% Pacotes adicionais, usados apenas no âmbito do Modelo Canônico do abnteX2
% ---
\usepackage{lipsum}				% para geração de dummy text
% ---

% ---
% Pacotes de citações
% ---
\usepackage[brazilian,hyperpageref]{backref}	 % Paginas com as citações na bibl
\usepackage[alf]{abntex2cite}	% Citações padrão ABNT

% --- 
% CONFIGURAÇÕES DE PACOTES
% --- 

% ---
% Configurações do pacote backref
% Usado sem a opção hyperpageref de backref
\renewcommand{\backrefpagesname}{Citado na(s) página(s):~}
% Texto padrão antes do número das páginas
\renewcommand{\backref}{}
% Define os textos da citação
\renewcommand*{\backrefalt}[4]{
	\ifcase #1 %
		Nenhuma citação no texto.%
	\or
		Citado na página #2.%
	\else
		Citado #1 vezes nas páginas #2.%
	\fi}%
% ---


% ---
% Informações de dados para CAPA e FOLHA DE ROSTO
% ---
\titulo{Paradigmas de Programação da linguagem LUA}
\autor{Mário Sérgio Oliveira de Queiroz \\ Pedro Martins}
\local{Brasil}
\data{25 de Novembro de 2013}
\orientador{João Paulo Ataíde Martins}
\instituicao{%
  IESB - Centro Universitário Instituto de Ensino Superior de Brasília
  \par
  Ciência da Computação}
\tipotrabalho{TCC (Graduação)}
% O preambulo deve conter o tipo do trabalho, o objetivo, 
% o nome da instituição e a área de concentração 
\preambulo{Projeto para a disciplina Projeto Integrador VI - Paradigmas de Linguagem de Programação, 
do Centro Universitário Instituto de Educação Superior de Brasília, DF.}
% ---


% ---
% Configurações de aparência do PDF final

% alterando o aspecto da cor azul
\definecolor{blue}{RGB}{41,5,195}

% informações do PDF
\makeatletter
\hypersetup{
     	%pagebackref=true,
		pdftitle={\@title}, 
		pdfauthor={\@author},
    	pdfsubject={\imprimirpreambulo},
	    pdfcreator={LaTeX with abnTeX2},
		pdfkeywords={abnt}{latex}{abntex}{abntex2}{trabalho acadêmico}, 
		colorlinks=true,       		% false: boxed links; true: colored links
    	linkcolor=blue,          	% color of internal links
    	citecolor=blue,        		% color of links to bibliography
    	filecolor=magenta,      		% color of file links
		urlcolor=blue,
		bookmarksdepth=4
}
\makeatother
% --- 

% --- 
% Espaçamentos entre linhas e parágrafos 
% --- 

% O tamanho do parágrafo é dado por:
\setlength{\parindent}{1.3cm}

% Controle do espaçamento entre um parágrafo e outro:
\setlength{\parskip}{0.2cm}  % tente também \onelineskip

% ---
% compila o indice
% ---
\makeindex
% ---

% ----
% Início do documento
% ----
\begin{document}

% Retira espaço extra obsoleto entre as frases.
\frenchspacing 

% ----------------------------------------------------------
% ELEMENTOS PRÉ-TEXTUAIS
% ----------------------------------------------------------
% \pretextual

% ---
% Capa
% ---
\imprimircapa
% ---

% ---
% Folha de rosto
% (o * indica que haverá a ficha bibliográfica)
% ---
\imprimirfolhaderosto*
% ---

% ---
% Inserir a ficha bibliografica
% ---

% Isto é um exemplo de Ficha Catalográfica, ou ``Dados internacionais de
% catalogação-na-publicação''. Você pode utilizar este modelo como referência. 
% Porém, provavelmente a biblioteca da sua universidade lhe fornecerá um PDF
% com a ficha catalográfica definitiva após a defesa do trabalho. Quando estiver
% com o documento, salve-o como PDF no diretório do seu projeto e substitua todo
% o conteúdo de implementação deste arquivo pelo comando abaixo:
%
% \begin{fichacatalografica}
%     \includepdf{fig_ficha_catalografica.pdf}
% \end{fichacatalografica}
\begin{fichacatalografica}
	\vspace*{\fill}					% Posição vertical
	\hrule							% Linha horizontal
	\begin{center}					% Minipage Centralizado
	\begin{minipage}[c]{12.5cm}		% Largura
	
	\imprimirautor
	
	\hspace{0.5cm} \imprimirtitulo  / \imprimirautor. --
	\imprimirlocal, \imprimirdata-
	
	\hspace{0.5cm} \pageref{LastPage} p. : il. (algumas color.) ; 30 cm.\\
	
	\hspace{0.5cm} \imprimirorientadorRotulo~\imprimirorientador\\
	
	\hspace{0.5cm}
	\parbox[t]{\textwidth}{\imprimirtipotrabalho~--~\imprimirinstituicao,
	\imprimirdata.}\\
	
	\hspace{0.5cm}
		1. Palavra-chave1.
		2. Palavra-chave2.
		I. Orientador.
		II. Universidade xxx.
		III. Faculdade de xxx.
		IV. Título\\ 			
	
	\hspace{8.75cm} CDU 02:141:005.7\\
	
	\end{minipage}
	\end{center}
	\hrule
\end{fichacatalografica}
% ---

% ---
% Inserir errata
% ---
\begin{errata}
Elemento opcional da \citeonline[4.2.1.2]{NBR14724:2011}. Exemplo:

\vspace{\onelineskip}

FERRIGNO, C. R. A. \textbf{Tratamento de neoplasias ósseas apendiculares com
reimplantação de enxerto ósseo autólogo autoclavado associado ao plasma
rico em plaquetas}: estudo crítico na cirurgia de preservação de membro em
cães. 2011. 128 f. Tese (Livre-Docência) - Faculdade de Medicina Veterinária e
Zootecnia, Universidade de São Paulo, São Paulo, 2011.

\begin{table}[htb]
\center
\footnotesize
\begin{tabular}{|p{1.4cm}|p{1cm}|p{3cm}|p{3cm}|}
  \hline
   \textbf{Folha} & \textbf{Linha}  & \textbf{Onde se lê}  & \textbf{Leia-se}  \\
    \hline
    1 & 10 & auto-conclavo & autoconclavo\\
   \hline
\end{tabular}
\end{table}

\end{errata}
% ---

% ---
% Inserir folha de aprovação
% ---

% Isto é um exemplo de Folha de aprovação, elemento obrigatório da NBR
% 14724/2011 (seção 4.2.1.3). Você pode utilizar este modelo até a aprovação
% do trabalho. Após isso, substitua todo o conteúdo deste arquivo por uma
% imagem da página assinada pela banca com o comando abaixo:
%
% \includepdf{folhadeaprovacao_final.pdf}
%
\begin{folhadeaprovacao}

  \begin{center}
    {\ABNTEXchapterfont\large\imprimirautor}

    \vspace*{\fill}\vspace*{\fill}
    {\ABNTEXchapterfont\bfseries\Large\imprimirtitulo}
    \vspace*{\fill}
    
    \hspace{.45\textwidth}
    \begin{minipage}{.5\textwidth}
        \imprimirpreambulo
    \end{minipage}%
    \vspace*{\fill}
   \end{center}
    
   Trabalho aprovado. \imprimirlocal, \imprimirdata:

   \assinatura{\textbf{\imprimirorientador} \\ Orientador} 
   \assinatura{\textbf{Professor} \\ Convidado 1}
   \assinatura{\textbf{Professor} \\ Convidado 2}
      
   \begin{center}
    \vspace*{0.5cm}
    {\large\imprimirlocal}
    \par
    {\large\imprimirdata}
    \vspace*{1cm}
  \end{center}
  
\end{folhadeaprovacao}
% ---

% ---
% Dedicatória
% ---
\begin{dedicatoria}
   \vspace*{\fill}
   \centering
   \noindent
   \textit{ Este trabalho é dedicado às crianças adultas que,\\
   quando pequenas, sonharam em se tornar cientistas.} \vspace*{\fill}
\end{dedicatoria}
% ---

% ---
% Agradecimentos
% ---
\begin{agradecimentos}
Os agradecimentos principais são direcionados à Gerald Weber, Miguel Frasson,
Leslie H. Watter, Bruno Parente Lima, Flávio de Vasconcellos Corrêa, Otavio Real
Salvador, Renato Machnievscz\footnote{Os nomes dos integrantes do primeiro
projeto abn\TeX\ foram extraídos de
\url{http://codigolivre.org.br/projects/abntex/}} e todos aqueles que
contribuíram para que a produção de trabalhos acadêmicos conforme
as normas ABNT com \LaTeX\ fosse possível.

Agradecimentos especiais são direcionados ao Centro de Pesquisa em Arquitetura
da Informação\footnote{\url{http://www.cpai.unb.br/}} da Universidade de
Brasília (CPAI), ao grupo de usuários
\emph{latex-br}\footnote{\url{http://groups.google.com/group/latex-br}} e aos
novos voluntários do grupo
\emph{\abnTeX}\footnote{\url{http://groups.google.com/group/abntex2} e
\url{http://abntex2.googlecode.com/}}~que contribuíram e que ainda
contribuirão para a evolução do \abnTeX.

\end{agradecimentos}

% ---
% Epígrafe
% ---
\begin{epigrafe}
    \vspace*{\fill}
	\begin{flushright}
		\textit{``Não vos amoldeis às estruturas deste mundo, \\
		mas transformai-vos pela renovação da mente, \\
		a fim de distinguir qual é a vontade de Deus: \\
		o que é bom, o que Lhe é agradável, o que é perfeito.\\
		(Bíblia Sagrada, Romanos 12, 2)}
	\end{flushright}
\end{epigrafe}
% ---

% ---
% RESUMOS
% ---

% resumo em português
\begin{resumo}
 Segundo a \citeonline[3.1-3.2]{NBR6028:2003}, o resumo deve ressaltar o
 objetivo, o método, os resultados e as conclusões do documento. A ordem e a extensão
 destes itens dependem do tipo de resumo (informativo ou indicativo) e do
 tratamento que cada item recebe no documento original. O resumo deve ser
 precedido da referência do documento, com exceção do resumo inserido no
 próprio documento. (\ldots) As palavras-chave devem figurar logo abaixo do
 resumo, antecedidas da expressão Palavras-chave:, separadas entre si por
 ponto e finalizadas também por ponto.

 \vspace{\onelineskip}
    
 \noindent
 \textbf{Palavras-chaves}: latex. abntex. editoração de texto.
\end{resumo}

% resumo em inglês
\begin{resumo}[Abstract]
 \begin{otherlanguage*}{english}
   This is the english abstract.

   \vspace{\onelineskip}
 
   \noindent 
   \textbf{Key-words}: latex. abntex. text editoration.
 \end{otherlanguage*}
\end{resumo}

% ---
% inserir lista de ilustrações
% ---
\pdfbookmark[0]{\listfigurename}{lof}
\listoffigures*
\cleardoublepage
% ---

% ---
% inserir lista de tabelas
% ---
\pdfbookmark[0]{\listtablename}{lot}
\listoftables*
\cleardoublepage
% ---

% ---
% inserir lista de abreviaturas e siglas
% ---
\begin{siglas}
  \item[Fig.] Area of the $i^{th}$ component
  \item[456] Isto é um número
  \item[123] Isto é outro número
  \item[lauro cesar] este é o meu nome
\end{siglas}
% ---

% ---
% inserir lista de símbolos
% ---
\begin{simbolos}
  \item[$ \Gamma $] Letra grega Gama
  \item[$ \Lambda $] Lambda
  \item[$ \zeta $] Letra grega minúscula zeta
  \item[$ \in $] Pertence
\end{simbolos}
% ---

% ---
% inserir o sumario
% ---
\pdfbookmark[0]{\contentsname}{toc}
\tableofcontents*
\cleardoublepage
% ---

% ----------------------------------------------------------
% ELEMENTOS TEXTUAIS
% ----------------------------------------------------------
\textual

% ----------------------------------------------------------
% Introdução
% ----------------------------------------------------------
\chapter{Introdução}
Este trabalho acadêmico se refere ao desenvolvimento de um estudo e pesquisa, relativo aos paradigmas e conceitos da linguagem de programação Lua. Desta forma, serão atingidos temas como implementação de de sintaxe e semântica da linguagem.

\section{Motivação}
Conforme a proposta de projeto para o semestre, no que se refere ao estudo dos paradigmas de uma linguagem de programação, a escolha do grupo pela linguagem LUA, teve vários estímulos, como o fato da linguagem ter surgido em uma universidade brasileira, além de possuir uma ampla aplicação no ambiente de jogos e na indústria de TV digital.

Em virtude do que foi mencionado, existiram muitas influênciasa para a escolha de LUA para esse projeto, existia o interesse em outras linguagens como Python, más devido as outras escolhas, optamos por LUA, que inclusive temos algumas experiências de trabalho.

\section{Objetivos}

\subsection{Geral}
Este trabalho tem como objetivo aplicar os conhecimentos obtidos na disciplina de paradigmas de linguagem de programação à linguagem LUA. Aprodundar e colocar em prática os conceitos aprendidos em sala de aula, documentando e exemplificando o funcionamento da linguagem LUA.

\subsection{Específicos}
Com base no objetivo geral derivam-se os seguintes objetivos específicos.
\begin{itemize}
\item Embasar historicamente a lingaguem.
\item Explicar o funcionamento da sintaxe e semântica.
\item Demonstrar os paradígmas envolvidos na linguagem.
\item Explicar e exemplificar o funcionamento de variáveis. Incluindo os tipos, sua vinculação, verificação de tipo e escopo.
\item Apresentar as vantagens, desvantagens e as áreas a qual LUA melhor se aplica.
\item Criação códigos para exemplificar os conceitos apresentados.
\end{itemize}

\section{Organização do Trabalho}


% ----------------------------------------------------------
% Capitulo com exemplos de comandos inseridos de arquivo externo 
% ----------------------------------------------------------

\include{abntex2-modelo-include-comandos}

% ---
% Capitulo de revisão de literatura
% ---
\chapter{Histórico}
A linguagem Lua foi totalmente projetada, e implementada no Brasil, por Roberto Ierusalimschy, Luiz Henrique de Figueiredo e Waldemar Celes, que eram membros do Computer Graphics Technology Group na PUC-Rio, a Pontifícia Universidade Católica do Rio de Janeiro. Lua nasceu e cresceu no Tecgraf, Grupo de Tecnologia em Computação Gráfica da PUC-Rio. Atualmente, Lua é desenvolvida no laboratório Lablua. Tanto o Tecgraf quanto Lablua são laboratórios do Departamento de Informática da PUC-Rio.

O estímulo inicial para a construção da linguagem veio de um projeto entre a PETROBRAS e a PUC-RIO, a fim de produzir um programa de interfaces gráficas para várias aplicações.

\begin{figure}[!htb]
	\centering
	\includegraphics[width=0.5\linewidth]{imagens/imagem1.png}
	\caption{Programa Gráfico Mestre}
\end{figure}

Logo surgiu o primeiro protótipo, DEL - Linguagem para Especificação de Diálogos, que trabalhava com lista de parâmetros e tipos e valores padrões. Com o passar do tempo após pesquisas e mudanças no projeto surgiu a linguagem `SOL' - Simple Object Language, sendo que era uma linguagem para descrição de objetos, inspirada no bibTex.

\begin{figure}[!htb]
	\centering
	\includegraphics[width=0.5\linewidth]{imagens/imagem2.png}
	\caption{Trecho de código da linguagem SOL}
\end{figure}

No entanto, tanto DEL como SOL tinha várias limitações como, pouco recurso para construção de diálogos, pouca abstração de dados e incompleta se comparadas às linguagens contemporâneas a elas. Então Roberto Ierusalimschy (PGM), Luiz Henrique de Figueiredo (DEL) e Waldemar Celes (PGM) se juntaram para achar uma solução comum a seus problemas. As propostas de solução era formular uma nova linguagem de configuração genérica, que fosse facilmente acoplável, portátil, simples e uma sintaxe fácil. Para o resultado desse projeto foi dado o nome LUA, como um contráste da antiga SOL. 

As linguagens que mais se aproximam das características de Lua são o Icon, por sua concepção, e Python, por sua facilidade de utilização. Em um artigo publicado no Dr. Dobb's Journal, os criadores de Lua também afirmam que Lisp e Scheme foram uma grande influência na decisão de desenvolver a tabela como a principal estrutura de dados de Lua. Lua tem sido usada em várias aplicações, tanto comerciais como não-comerciais.

Versões de Lua antes da versão 5.0 foram liberadas sob uma licença similar à licença BSD. A partir da versão 5.0, Lua foi licenciada sob a licença MIT.

Hoje a linguagem é uma das mais utilizads do mundo estando entre as vinte mais utilizadas.

\chapter{Aspectos léxicos e sintáticos de Lua}
Esta capítulo descreve os principais aspectos léxicos, sintáticos e semânticos da linguagem Lua. sendo assim, serão descritas quais itens léxicos são válidos, como eles são combinados, e qual o significado da sua combinação.

O estudo de linguagens de programação pode ser orientado à verificação dos aspectos semânticos e sintáticos de uma linguagem. Pois a sintaxe, é a forma das expressões e instruções, ou seja, como é feita a construção das mesmas. Não obstante, a semântica é o significado das expressões e instruções.

\section{Convenções Léxicas}
Em Lua, os nomes podem ser qualquer cadeia de letras, dígitos, e sublinhados que não começam com um dígito, assim como em outras linguagens tradicionais, como C/C++. Os identificadores são usados para nomear variáveis e campos de tabelas.

Lua é uma linguagem que diferencia letras minúsculas de maiúsculas, por exemplo, and é uma palavra reservada, mas And e AND são dois nomes válidos diferentes. Como convenção, nomes que começam com um sublinhado seguido por letras maiúsculas são reservados para variáveis globais internas usadas por Lua.

As seguintes cadeias denotam outros itens léxicos:
+ - * == ~= <= >= < > = ( ) { } [ ] ; : , . .. ...

As cadeias de caracteres literais podem ser delimitadas através do uso de aspas simples ou aspas duplas, e podem conter as seguintes seqüências de escape no estilo de C:  `contra-barra + a' (campainha), `contra-barra + b' (backspace), `contra-barra + f' (alimentação de formulário), `contra-barra + n' (quebra de linha), `contra-barra + r' (retorno de carro), `contra-barra + t' (tabulação horizontal), `contra-barra + v' (tabulação vertical), `contra-barra + contra-barra' (barra invertida), `contra-barra + aspas duplas' (citação [aspa dupla]) e `contra-barra + aspas simples'. Além disso, uma barra invertida seguida por uma quebra de linha real resulta em uma quebra de linha na cadeia de caracteres. Um caractere em uma cadeia de caracteres também pode ser especificado pelo seu valor numérico usando a seqüência de escape contra-barra + ddd, onde ddd é uma seqüência de até três dígitos decimais. (Note que se um caractere numérico representado como um seqüência de escape for seguido por um dígito, a seqüência de escape deve possuir exatamente três dígitos.) Cadeias de caracteres em Lua podem conter qualquer valor de 8 bits, incluindo zeros dentro delas, os quais podem ser especificados como `contra-barra + 0'.

Cadeias literais longas podem ser definidas usando um formato longo delimitado por colchetes. Definimos uma abertura de colchete longo de nível n como um abre colchete seguido por n sinais de igual seguido por outro abre colchete. Dessa forma, uma abertura de colchete longo de nível 0 é escrita como [[, uma abertura de colchete longo de nível 1 é escrita como [=[ e assim por diante. Um fechamento de colchete longo é definido de maneira similar. Uma cadeia de caracteres longa começa com uma abertura de colchete longo de qualquer nível e termina no primeiro fechamento de colchete longo do mesmo nível. Literais expressos desta forma podem se estender por várias linhas, não interpretam nenhuma seqüência de escape e ignoram colchetes longos de qualquer outro nível. Estes literais podem conter qualquer coisa, exceto um fechamento de colchete longo de nível igual ao da abertura.

As seguintes palavras-chave são reservadas e não podem ser utilizadas como nomes:

and;

break;

do;

else e elseif;

end;

false;

for;

function;

if;

in;

local;

nil;

not;

or;

repeat;

return;

then;

true;

until;

while.

\section{Sintaxe de Lua}
Aqui está a sintaxe completa de Lua na notação BNF estendida. (Ela não descreve as precedências dos operadores.)

\begin{figure}[!htb]
	\centering
	\includegraphics[width=0.9\linewidth]{imagens/sintaxe1.png}
\end{figure}

\begin{figure}[!htb]
	\centering
	\includegraphics[width=0.9\linewidth]{imagens/sintaxe2.png}
\end{figure}


% ---
% segundo capitulo de Resultados
% ---
\chapter{Semântica das Variáveis}
Este capítulo apresenta as questões fundamentais das variáveis. Como tipo, endereço e valores. Além da abordagem de vinculação e de escopo.

\section{Variáveis}
Uma variável em uma linguagem é a abstração do conteúdo de células de memória do computador. Variáveis podem se caracterizadas de acordo com os seguintes aspectos: nome, endereço, valor, tipo, tempo de vida e escopo.

Em Lua existem três tipos de variáveis, sendo elas as seguites: variáveis globais, variáveis locais e variáveis de tabelas. Sendo que, a diferença entre variáveis locais e globais é o uso da palavra reservada `local' antes do nome da variável. Já as variáveis de tabela são os nomes dados aos índices das tabelas, tendo em vista que toda a estrutura de dados linguagem é orientada à tabelas.

\section{Vinculação}
O termo vinculação é uma associação ou uma referência, como, por exemplo, entre uma atributo e uma entidade e entre uma operação e um símbolo. O momento em que ocorre a vinculação é denominado como tempo de vinculação. Isso porquê, as vinculações podem ocorrer no tempo de projeto da linguagem, no tempo de implementação, no tempo de compilação, no tempo de ligação, no tempo de carregamento ou no tempo de execução. Um bom exemplo disso é que o operador `+' é vinculado no tempo de projeto da linguagem.

Lua é uma linguagem dinamicamente tipada. Isto significa que variáveis não possuem tipos, porém somente valores possuem tipos. Não existem definições de tipos na linguagem, pois todos os valores carregam o seu próprio tipo de dados. Logo Lua utilizaça um método de decleração implícita de variáveis.

A linguagem trabalha com vinculação dinâmica de tipos, logo o tipo não é especificado por uma instrução de declaração, como em C ou em JAVA. Em vez disso, a variável é vinculada a um tipo quando lhe é atribuida um valor em uma instrução de atribuição.

Este modelo apresenta muitas diferenças com relação aos tipos estaticamente vinculados. A principal vantagem de vinculação dinâmica de variáveis a tipos é que ele tráz muita flexibilidade para a programação.

Existem oito tipos de dados básicos em Lua, são eles, nil, boolean, number, string, function, userdata, thread e table. Nil é o tipo do valor nulo, cuja propriedade principal é ser diferente de qualquer outro valor, ele geralmente representa a ausência de um valor útil. Boolean é o tipo dos valores false e true. Tanto nil como false tornam uma condição falsa, sendo que, qualquer outro valor torna a condição verdadeira. Number representa números reais (ponto flutuante de precisão dupla). O tipo string representa cadeias de caracteres.

O tipo userdata permite que dados C arbitrários possam ser armazenados em variáveis Lua. Este tipo corresponde a um bloco de memória e não tem operações pré-definidas em Lua. O tipo thread representa fluxos de execução independentes e é usado para implementar co-rotinas. Não se pode confundir o tipo thread de Lua com os processos leves do sistema operacional, pois Lua dá suporte a co-rotinas em todos os sistemas.

O tipo table implementa arrays associativos, isto é, arrays que podem ser indexados não apenas por números, mas por qualquer valor (exceto nil). Tabelas podem ser heterogêneas, ou seja, elas podem conter valores de todos os tipos (exceto nil). Tabelas são o único mecanismo de estruturação de dados em Lua, todavia, elas podem ser usadas para representar arrays comuns, tabelas de símbolos, conjuntos, registros, grafos, árvores, etc.

\section{Verificação de Tipos}
A verificação de tipos é um módulo que assegura que os operandos de um operador sejam de tipos compatíveis.

\section{Escopo}
O escopo de uma variável em uma linguagem é a faixa de instruções na qual a variável é visível. Uma variável é visível em uma instrução se puder ser referenciada nessa instrução, ou seja se uma instrução conseguer ter acesso a essa variável no bloco ou setor em que se encontra.

Tendo em vista que existem dois tipos de escopo, sendo eles o escopo estático e o escopo dinâmico, Lua trabalha na modelagem de escopo dinâmico, logo baseia-se na sequência de chamadas de subprogramas. Dessa forma, o escopo pode ser determidado em tempo de execução.

Assume-se que toda variável é uma variável global a menos que ela seja explicitamente declarada como uma variável local. Variáveis locais possuem escopo léxico, por isso podem ser livremente acessadas por funções definidas dentro do seu escopo ou bloco.

Um bloco é uma lista de comandos; sintaticamente, um bloco é a mesma coisa que um trecho. Sendo que, o bloco pode ser explicitamente delimitado para produzir um único comando:

-comando ::= do bloco end

Blocos explícitos são úteis para controlar o escopo de declarações de variáveis, além de também usados às vezes para adicionar um comando return ou break no meio de outro bloco.

Lua é uma linguagem com escopo léxico. O escopo das variáveis começa no primeiro comando depois da sua declaração e vai até o fim do bloco mais interno que inclui a declaração. Considere o seguinte exemplo:

\begin{figure}[!htb]
	\centering
	\includegraphics[width=0.5\linewidth]{imagens/imagem3.png}
	\caption{Verificação de escopo em Lua}
\end{figure}

Note que, em uma declaração como local x = x, o novo x sendo declarado não está no escopo ainda e portanto o segundo x se refere a uma variável externa.

Por causa das regras de escopo léxico, variáveis locais podem ser livremente acessadas por funções definidas dentro do seu escopo. Uma variável local usada por uma função mais interna é chamada de upvalue ou variável local externa, dentro da função mais interna.


\bookmarksetup{startatroot} 
% --- 

% ---
% Conclusão
% ---
\chapter{Conclusão}
\cite{Sebesta}
\cite{Valentim:2013:Online}


% ----------------------------------------------------------
% ELEMENTOS PÓS-TEXTUAIS
% ----------------------------------------------------------
\postextual


% ----------------------------------------------------------
% Referências bibliográficas
% ----------------------------------------------------------
\bibliography{referencia}


% ----------------------------------------------------------
% Apêndices
% ----------------------------------------------------------

% ---
% Inicia os apêndices
% ---
\begin{apendicesenv}

% ----------------------------------------------------------
\chapter{apendice 1}
% ----------------------------------------------------------

paradgmas

% ----------------------------------------------------------
\chapter{apendice 2}
% ----------------------------------------------------------

paradgmas

\end{apendicesenv}
% ---


% ----------------------------------------------------------
% Anexos
% ----------------------------------------------------------

% ---
% Inicia os anexos
% ---
\begin{anexosenv}


\chapter{anexo 1}


\chapter{anexo 1}


\chapter{anexo 1}


\end{anexosenv}

%---------------------------------------------------------------------
% INDICE REMISSIVO
%---------------------------------------------------------------------

\printindex

\end{document}
